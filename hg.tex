\section{Formally Verified Disassembly}
\begin{frame}{Breakdown of Contribution}
  \begin{outline}[enumerate] % Don't think this environment works with <+-> directly
    \1<+-> \alert{Formal models} for
      \2 State predicates
      \2 Memory relations
    \1<+-> \alert{Sound algorithm} to lift \glsxtrfullpl{hg} from binaries
    \1<+-> \alert{Evaluation} showing coverage and indirection resolution
  \end{outline}
\end{frame}

\subsection{Example}

\begin{frame}[fragile]{Full Example}{A \emph{Hoare Graph}}
  \begin{columns}
    \column{.48\textwidth}
    \begin{block}{32-bit \gls{x86} for display purposes}
      \begin{lstlisting}[style=x64,gobble=8,basicstyle=\small\ttfamily]
        0x0 : cmp eax,c3 |\label{hg-example-cmp}|
        0x5 : ja  1c     |\label{hg-example-ja}|
        0xb : mov eax,DWORD PTR [eax*4+a]|\label{hg-example-jump-table-read}|
        0x12: mov DWORD PTR [edi],eax |\label{hg-example-mov1}|
        0x14: mov DWORD PTR [esi],1   |\label{hg-example-mov2}|
        0x1a: jmp DWORD PTR [edi]     |\label{hg-example-indirect-jump}|
      \end{lstlisting}
    \end{block}

    \column{.52\textwidth}
    % combination insser sep/minimum size ensures all nodes have exactly same size
    \tikzset{vertex/.style = {circle,draw,minimum size=0.5cm}} % inner sep=0pt, <- what does this even do?
    \begin{tikzpicture}[font=\footnotesize] % if time permits play around with setting -> here
      \node[vertex]                   (0)     {$\mathtt{0}$};
      \node[vertex,right=.5cm of 0]        (5)     {$\mathtt{5}$};
      \node[vertex,above right=.1cm and .5cm of 5]  (1c)    {$\mathtt{1c}$};
      \node[draw=none,right=2cm of 1c]    (1cret) {};
      \node[vertex,below right=.1cm and .5cm of 5]  (b)     {$\mathtt{b}$};
      \node[vertex,below=.2cm of b]        (122)   {$\mathtt{12}$};
      \node[draw=none,below left=.5cm of b]    (121)   {};
      \node[draw=none,left=.1cm of 121]    (120)   {};
      \node[draw=none,below right=.5cm of b]   (123)   {};
      \node[draw=none,right=.1cm of 123]   (124)   {};
      \node[vertex,below=.2cm of 122]      (14)    {$\mathtt{14}$};
      \node[vertex,below right=.5cm of 14] (1a2)   {$\mathtt{1a}$};
      \node[vertex,below=.2cm of 1a2]      (ptr)   {$a_\mathtt{jt}$};
      \node[draw=none,right=of ptr]   (ptret) {};
      \node[vertex,below left=.5cm of 14]  (1a1)   {$\mathtt{1a}$};
      \node[vertex,below=.2cm of 1a1]      (1)     {$\mathtt{1}$};
      \node[vertex,left=of 1]         (ret)   {$a_\mathtt{r}$};

      % right tells tikz to start drawing the node right of the position (instead of centered)
      \node[right,text width=3cm,align=left] at (-1.8,.6) {\begin{align*}
          P_0 &= \readmem{\reg{rsp}}4 == a_\mathtt{r}\\
          M_0 &= \varnothing
      \end{align*}};

      \node[right=.1cm of b] {$
        \reg{eax} < \mathtt{0xc3}
        $};

      \node[right=.1cm of 122] {$
        \reg{eax} == a_\mathtt{jt}
        $};

      \node[right=.01cm of 14] {$
        \readmem{\reg{edi}}4 == a_\mathtt{jt}
        $};

      \node[right=.01cm of 1a2] {$
        \begin{array}{c}
          \gls{separate} \\
          \cdots a_\mathtt{jt}
        \end{array}
        $};

      \node[left=.1cm of 1a1] {$
        \begin{array}{c}
          \gls{alias} \\
          \cdots1
        \end{array}
        $};

      \draw [overlay,decorate,decoration={brace,amplitude=10pt,mirror},xshift=-4pt] (4.5,-1.2) -- (4.5,-.5) node [black,midway,xshift=1cm] {
        \begin{tabular}{l}
          up to\\$\mathtt{0xc3}$\\
          edges:\\one\\
          per\\read\\
          value
        \end{tabular}
      };

      \path[->] (0) edge node [above] {\inlineasm{cmp}} (5);
      \path[->] (5) edge node [above] {\inlineasm{ja}} (1c);
      \path[->] (5) edge node [below] {\inlineasm{ja}} (b);
      \draw[dotted,thick,->] (1c) edge node [above] {$\reg{eax} \geq \mathtt{0xc3}$} (1cret);
      \draw[dotted,thick,->] (b)   to (120);
      \draw[dotted,thick,->] (b)   to (121);
      \draw[->]        (b)   to (122);
      \draw[dotted,thick,->] (b)   to (123);
      \path[dotted,thick,->] (b)   edge node [right,xshift=0.2] {\inlineasm{mov}} (124);
      \path[->]        (122) edge node [right] {\inlineasm{mov}} (14);
      \path[->]        (1a2) edge node [right] {\inlineasm{jmp}} (ptr);
      \draw[dotted,thick,->] (ptr) to (ptret);
      \path[->]        (14)  edge node [left]  {\inlineasm{mov}} (1a1);
      \path[->]        (14)  edge node [right] {\inlineasm{mov}} (1a2);
      \path[line width=2pt,->] (1a1) edge node [left] {\inlineasm{jmp}} (1);
      \path[line width=2pt,->] (1) edge node [below] {{\textbf{\inlineasm{ret}}}} (ret);
    \end{tikzpicture}
  \end{columns}
\end{frame}

\begin{frame}{State of the Art}
  \begin{block}{Jakstab}
    \begin{outline}
      \1 Uses \alert{abstract interpretation}
      \1 Requires significant \alert{test harnesses}
      \1 Does not achieve full \alert{overapproximation}
    \end{outline}
  \end{block}
\end{frame}


\subsection{Formulation}

\begin{frame}{Base Algorithm}
  \begin{outline}[enumerate]
    \1<+-> Pop state off \gls{bag} of states to analyze
    \1<+-> If state \alert{``less than''} existing state in \gls{hg}, join and update \gls{hg}
    \1<+-> Evaluate current instruction in potentially-joined state; may result in multiple result states
    \1<+-> Can continue? For each result:
      \2 If yes, add to bag
      \2 If no, annotate in \gls{hg}
  \end{outline}
\end{frame}

\begin{frame}{Extension for Function Calls}
  \begin{block}{State Predicate Modifications}
    \begin{outline}
      \1<+-> External calls preserve only stack, \alert{caller-saved} registers
      \1<+-> Internal calls
        \2 Preserve stack, \alert{caller-saved} registers at call site
        \2 State within call reset, allowing reuse of results
        \2 Perform join of state-within-call and post-call state for subcall return
      \1<+-> Requires concept of \alert{reachability}; post-call states not continued unless known reachable
    \end{outline}
  \end{block}
\end{frame}

% Leaving out validation as that was Freek's work. I'll answer questions if it gets brought up but dedicating time to it explicitly is probably not worth it.

\subsection{Evaluation}

\begin{frame}{Case Study: Xen Project}
  \begin{columns}
    \column{.35\textwidth}
    \begin{center}
      \colorbox{black}{\includegraphics[width=.8\linewidth]{logo_xenproject}}\\[1em]
      \includegraphics[width=.5\linewidth]{aws_logo-100781597-large.3x2}
    \end{center}

    \column{.6\textwidth}
    \begin{block}{Properties}
      \begin{outline}
        \1 Widely used \gls{vmm}/hypervisor (e.g., Amazon AWS)
        \1 Mostly written in \gls{c}
      \end{outline}
    \end{block}
  \end{columns}
\end{frame}

% Fragile was needed with the newcolumn being inside the frame but moving it out fixed that
% number glossaries with \glsentrysymbol to get unformatted numbers weren't working so just putting the numbers directly
\begin{frame}[label=hg-results]{Statistics Summarized}
  \centering
  \begin{tabular}{l
      S[table-format=4]@{$=$}
      S[table-format=4]@{$+$}
      S[table-format=2]@{$+$}
      S[table-format=2]@{$+$}
      S[table-format=1]
      S[table-format=6]
      S[table-format=6]
      S[table-format=2]
      S[table-format=3]
      S[table-format=3]
      r}
    \toprule
    \thead{Binary Target Type} & {\thead{Total}} & {\thead{$w$}} & {\thead{$x$}} & {\thead{$y$}} & {\thead{$z$}} & {\thead{Instrs.}} & {\thead{Symbolic\\States}} & {\thead{A}} & {\thead{B}} & {\thead{C}} & \thead{Time/\\h:m:s} \\
    \midrule
    Programs & 63 & 316 & 3 & 13 & 1 & 18124 & 18562 & 56 & 26 & 11 & 0:35:59 \\
    Library Functions & 2151 & 2115 & 32 & 0 & 4 & 381647 & 391524 & 1 & 244 & 720 & 17:35:42 \\
    \bottomrule
  \end{tabular}\\
  \begin{tabular}{rcl rcl rcl}
    \multicolumn{9}{c}{$w$ lifted, $x$ unprovable return address, $y$ concurrency, $z$ timeout} \\
    A &=& Resolved indirection & B &=& Unresolved jump(s) & C &=& Unresolved call(s) \\
  \end{tabular}
  \hyperlink{timing}{\beamerskipbutton{Timing}}
\end{frame}

\begin{frame}{Discussion and Limitations}
  \begin{columns}
    \column{.35\textwidth}
    \begin{block}{Uses}
      \begin{outline}
        \1 Security analyses
        \1 Verification/\gls{tcb} reduction
        \1 Decompilation
        \1 Patching
      \end{outline}
    \end{block}

    \pause
    \column{.52\textwidth}
    \begin{block}{Limitations}
      \begin{outline}
        \1 Contextuality for return addresses, indirection
        \1 No concurrency support
        \1 No \gls{cpp} \gls{eh} support
          \2 Not an issue for Xen, but\dots
      \end{outline}
    \end{block}
  \end{columns}
\end{frame}
