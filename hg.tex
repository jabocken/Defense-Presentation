\section{Hoare Graphs}

\subsection{Example}

\begin{frame}[fragile]{Assembly Snippet}
  \begin{block}{32-bit \gls{x86} for display purposes}
    \begin{lstlisting}[style=x64,gobble=6]
      0x0 : 3dc3000000    cmp  eax,c3 |\label{hg-example-cmp}|
      0x5 : 0f8718000000  ja   1c     |\label{hg-example-ja}|
      0xb : 8b0485__a___  mov  eax,DWORD PTR [eax*4+a] |\label{hg-example-jump-table-read}|
      0x12: 8907          mov  DWORD PTR [edi],eax |\label{hg-example-mov1}|
      0x14: c70601000000  mov  DWORD PTR [esi],1   |\label{hg-example-mov2}|
      0x1a: ff27          jmp  DWORD PTR [edi]     |\label{hg-example-indirect-jump}|
    \end{lstlisting}
  \end{block}
\end{frame}

\begin{frame}{Hoare Graph Snippet}
  \centering
  % combination insser sep/minimum size ensures all nodes have exactly same size
  \tikzset{vertex/.style = {shape=circle,draw,minimum size=0.7cm}} % inner sep=0pt, <- what does this even do?
  \tikzset{edge/.style = {->,> = latex'}}
  \begin{tikzpicture}
    \node[vertex]    (0)     at  (0,0)  {$\mathtt{0}$};
    \node[vertex]    (5)     at  (1.5,0)  {$\mathtt{5}$};
    \node[vertex]    (1c)    at  (3,.5) {$\mathtt{1c}$};
    \node[draw=none] (1cret) at  (6,.5)  {};
    \node[vertex]    (b)     at  (3,-.5)  {$\mathtt{b}$};
    \node[draw=none] (120)   at  (1.6,-1.5)  {};
    \node[draw=none] (121)   at  (2.3,-1.5)  {};
    \node[vertex]    (122)   at  (3,-2)  {$\mathtt{12}$};
    \node[draw=none] (123)   at  (3.7,-1.5)  {};
    \node[draw=none] (124)   at  (4.4,-1.5)  {};
    \node[vertex]    (14)    at  (3,-3.5)  {$\mathtt{14}$};
    \node[vertex]    (1a2)   at  (3.5,-4.75)  {$\mathtt{1a}$};
    \node[vertex]    (ptr)   at  (3.5,-6)  {$a_\mathtt{jt}$};
    \node[draw=none] (ptret) at  (5,-6)  {};
    \node[vertex]    (1a1)   at  (2.5,-4.75)  {$\mathtt{1a}$};
    \node[vertex]    (1)     at  (2.5,-6)  {$\mathtt{1}$};
    \node[vertex]    (ret)   at  (1,-6)  {$a_\mathtt{r}$};

    % right tells tikz to start drawing the node right of the position (instead of centered)
    \node[right,text width=3.87cm,align=left] at (-2.5,.75) {\begin{align*}
        P_0 &= \readmem{\reg{rsp}} == a_\mathtt{r}\\
        M_0 &= \emptyset
    \end{align*}};

    \node[right] at (3.4,-.45) {$
      \reg{eax} < \mathtt{0xc3}
      $};

    \node[right] at (3.4,-2) {$
      \reg{eax} == a_\mathtt{jt}
      $};

    \node[right] at (3.4,-3.5) {$
      \readmem{\reg{edi}}4 == a_\mathtt{jt}
      $};

    \node[right] at (3.8,-4.75) {$
      \begin{array}{c}
        \region{\reg{edi}}4 \gls{separate} \region{\reg{esi}}4 \\
        \readmem{\reg{edi}}4 == a_\mathtt{jt}
      \end{array}
      $};

    \node[left] at (2.15,-4.75) {$
      \begin{array}{c}
        [\reg{edi},4] \equiv [\reg{esi},4] \\
        *[\reg{edi},4] == 1
      \end{array}
      $};

    \draw [overlay,decorate,decoration={brace,amplitude=10pt,mirror},xshift=-4pt] (4.75,-1.75) -- (4.75,-.5) node [black,midway,xshift=.7cm] {
      \begin{tabular}{l}
        up to $\mathtt{0xc3}$\\
        edges: one\\
        per read\\
        value
      \end{tabular}
    };

    \path[->] (0) edge node [above] {\inlineasm{cmp}} (5);
    \path[->] (5) edge node [above] {\inlineasm{ja}} (1c);
    \path[->] (5) edge node [below] {\inlineasm{ja}} (b);
    \draw[dotted,->] (1c) edge node [above] {$\reg{eax} \geq \mathtt{0xc3}$} (1cret);
    \draw[dotted,->] (b)   to (120);
    \draw[dotted,->] (b)   to (121);
    \draw[->]        (b)   to (122);
    \draw[dotted,->] (b)   to (123);
    \path[dotted,->] (b)   edge node [right,xshift=0.2] {\inlineasm{mov}} (124);
    \path[->]        (122) edge node [right] {\inlineasm{mov}} (14);
    \path[->]        (1a2) edge node [right] {\inlineasm{jmp}} (ptr);
    \draw[dotted,->] (ptr) to (ptret);
    \path[->]        (14)  edge node [left]  {\inlineasm{mov}} (1a1);
    \path[->]        (14)  edge node [right] {\inlineasm{mov}} (1a2);
    \path[line width=5pt,->] (1a1) edge node [left] {\inlineasm{jmp}} (1);
    \path[line width=5pt,->] (1) edge node [below] {{\textbf{\inlineasm{ret}}}} (ret);
  \end{tikzpicture}
\end{frame}


\subsection{Formulation}

% Leaving out validation as that was Freek's work. I'll answer questions if it gets brought up but dedicating time to it explicitly is probably not worth it.

\subsection{Results}

\begin{frame}{Case Study: Xen Project}
  \begin{columns}
    \column{.35\textwidth}
    \colorbox{black}{\includegraphics[width=\linewidth-2\fboxsep]{logo_xenproject}}

    \column{.6\textwidth}
    \begin{block}{Properties}
      \begin{outline}
        \1 Widely used \alert{\gls{vmm}/hypervisor}
        \1 Mostly written in \gls{c}
      \end{outline}
    \end{block}
  \end{columns}
\end{frame}

% Fragile was needed with the newcolumn being inside the frame but moving it out fixed that
\newcolumntype{C}[1]{>{\centering\let\newline\\\arraybackslash\hspace{0pt}}m{#1}}
\begin{frame}{Case Study Statistics Summary (Binaries)}
  \centering
  \begin{tabular}{lC{4.8ex}@{$=$}C{4.8ex}@{$+$}C{2.4ex}@{$+$}C{2.4ex}@{$+$}rrrrrrr}
    \toprule
    \thead{Directory} & \multicolumn{5}{c}{} & {\thead{Instrs.}} & {\thead{Symbolic\\States}} & {\thead{A}} & {\thead{B}} & {\thead{C}} & \thead{Time/\\h:m:s} \\
    \midrule
    \texttt{bin} & 15 & 12 & 2 & 1 & 0 & 6751 & 6829 & 21 & 19 & 0 & 0:15:54 \\
    \texttt{xen/bin} & 17 & 7 & 1 & 8 & 1 & 2433 & 2468 & 8 & 3 & 3 & 0:01:17 \\
    \texttt{libexec} & 1 & 1 & 0 & 0 & 0 & 82 & 87 & 1 & 0 & 0 & 0:00:10 \\
    \texttt{sbin} & 30 & 25 & 1 & 4 & 0 & 8858 & 9178 & 26 & 4 & 8 & 0:18:39 \\
    \midrule
    Total & \glssymbol{bin-total} & \glssymbol{bin-success} & 3 & 13 & 1 & 18\,124 & 18\,562 & 56 & 26 & 11 & 0:35:59 \\
    \bottomrule
  \end{tabular}\\
  \begin{tabular}{rcl rcl rcl}
    \multicolumn{9}{c}{$w+x+y+z$: $w$ lifted, $x$ unprovable return address, $y$ concurrency, $z$ timeout} \\
    A &=& Resolved indirection & B &=& Unresolved jump(s) & C &=& Unresolved call(s) \\
  \end{tabular}
\end{frame}

\begin{frame}{Case Study Statistics Summary (Library Functions)}
  \centering
  \begin{tabular}{lC{4.8ex}@{$=$}C{4.8ex}@{$+$}C{2.4ex}@{$+$}C{2.4ex}@{$+$}rrrrrrr}
    \toprule
    \thead{Directory} & \multicolumn{5}{c}{} & {\thead{Instrs.}} & {\thead{Symbolic\\States}} & {\thead{A}} & {\thead{B}} & {\thead{C}} & \thead{Time/\\h:m:s} \\
    \midrule
    \texttt{lib} & 1907 & 1874 & 29 & 0 & \glssymbol{lib-func-timeout} & 353\,433 & 362\,635 & 1 & 244 & 600 & 15:28:17 \\
    \texttt{xenfsimage} & 109 & 106 & 3 & 0 & 0 & 17\,184 & 17\,683 & 0 & 0 & 27 & 1:58:36 \\
    \texttt{dist-packages} & 16 & 16 & 0 & 0 & 0 & 379 & 407 & 0 & 0 & 3 & 0:00:06 \\
    \texttt{lowlevel} & 119 & 119 & 0 & 0 & 0 & 10\,651 & 10\,799 & 0 & 0 & 90 & 0:08:43 \\
    \midrule
    Total & \glssymbol{lib-func-total} & \glssymbol{lib-func-success} & 32 & 0 & \glssymbol{lib-func-timeout} & 381\,647 & 391\,524 & 1 & 244 & 720 & 17:35:42 \\
    \bottomrule
  \end{tabular}\\
  \begin{tabular}{rcl rcl rcl}
    \multicolumn{9}{c}{$w+x+y+z$: $w$ lifted, $x$ unprovable return address, $y$ concurrency, $z$ timeout} \\
    A &=& Resolved indirection & B &=& Unresolved jump(s) & C &=& Unresolved call(s) \\
  \end{tabular}
\end{frame}

\begin{frame}{Timing}
  \centering
  \begin{tikzpicture}
    \begin{axis}[
        width=.95\textwidth,
        height=.75\textheight,
      date coordinates in=y,
      date ZERO=0-0-0,
      yticklabel=\hour:\minute,
      xlabel=Instruction Count,
      ylabel=Time/h:m,
      title=Distribution of instructions versus time
    ]
      \addplot[scatter, only marks] table [col sep=comma]{data/timing-hg.csv};
    \end{axis}
  \end{tikzpicture}
\end{frame}


\subsection{Discussion}
\frame{Not sure if this should be left out}
