\section{Prelim Recap}
\begin{frame}[label=floyd]{Floyd-Style Verification}
  \begin{columns}
    \column{.45\textwidth}
    \begin{block}{Recap}
      \begin{outline}
        \1 Approach using \glspl{cfg} deconstructed with generated \alert{cutpoints}
        \1 Generated proofs loaded in \gls{isabelle}[/HOL] and finished there
        \1 Used to verify several manually-isolated functions from \alert{the HermitCore unikernel}
        \1 Drawback: Manually intensive, proofs often require user input
      \end{outline}
    \end{block}

    \column{.45\textwidth}
    \begin{example}[Sample Graph]
      \todo{figure from dissertation}
    \end{example}
  \end{columns}
\end{frame}

\begin{frame}[label=hoare]{Hoare-Style Verification}
  \begin{columns}
    \column{.45\textwidth}
    \begin{block}{Recap}
      \begin{outline}
        \1 Approach using \gls{scf} deconstructed by \alert{Hoare rules}
        \1 Generated proofs loaded in \gls{isabelle}[/HOL] and finished there
        \2 Much less interaction required, only for loops and subcalls
        \1 Used to verify dozens of HermitCore functions, including ones with subcalls; no need for manual extraction
        \1 Drawback: \gls{scf} explosion can result from spaghetti code
      \end{outline}
    \end{block}

    \column{.45\textwidth}
    \begin{example}[Sample \Gls{scf}]
      \todo{figure from dissertation}
    \end{example}
  \end{columns}
\end{frame}
